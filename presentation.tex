\documentclass{beamer}
%\documentclass[notes=only]{beamer}

\usepackage[utf8]{inputenc}
 
\usetheme{Dresden}

\title[SOFTENG 370 Tutorial 1 (2019)] %optional
{Linux, Pointers and pthreads}
  
\author{Edward Zhang}
 
% \institute[UoA] % (optional)
% {
%   Department of ECSE\\
%   The University of Auckland
% }
 
\date[July 2019] % (optional)
{SOFTENG 370 T1}

\begin{document}
\frame{\titlepage}
\section{Running UNIX}
\subsection{Options}
\begin{frame}
  \frametitle{You need a UNIX system}
  Some ways to get a UNIX system to do this assignment
  \begin{itemize}
    \item Dual Boot Linux
    \item Run Linux in a Virtual Machine
    \item Run natively on macOS
    \item Run within Windows Subsystem for Linux (WSL)
    \item Run within Windows Subsystem for Linux 2 (WSL2)
  \end{itemize}
\end{frame}
\subsection{Caveats}
\begin{frame}
  \frametitle{On Virtual Machines}
  You can use any distro you want, but you'll probably be able to get more help when googling if you use one of the more popular desktop ones.
  \begin{itemize}
    \item Ubuntu (probably 18.04 LTS)
    \item Fedora Workstation (my personal preference)
    \item Debian
    \item Arch (great wiki, and u use arch btw), Manjaro if you actually want an installer
  \end{itemize}
\end{frame}
\begin{frame}
  \frametitle{Misc.}
  \begin{block}{Hypervisors}
    Oracle's VirtualBox is the usual free go-to. I personally prefer VMWare Player, feel free to give it a try. Parallels is a good option on macOS, but it's \$\$\$. Also try Hyper-V on Windows if you have Pro and already have it enabled, as it lets you keep other Windows features on (like Windows Sandbox or Core Isolation). It also supports one-click install of Ubuntu.
  \end{block}
  \begin{block}{Dual-booting}
    Beware you may be unable to dual-boot on some hardware, such as Surface Devices (drivers are a bit of a pain, especially on the book; check r/surfacelinux for more resources), or the 2019 MacBook Pro (can't even install, T2 chip NVMe storage support broken).
  \end{block}
\end{frame}
\end{document}